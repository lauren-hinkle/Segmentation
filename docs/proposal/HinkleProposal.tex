\documentstyle{article}

\title{592 Project Proposal \\ Efficient Graph Based Segmentation}
\author{Lauren Hinkle}

\begin{document}
\maketitle

\section{Efficient Graph Based Segmentation}
Efficient Graph Based Segmentation was published in 2004 in the International
Journal of Computer Vision.  It approaches the problem of image segmentation
by approaching an image as a graph, and attempts to optimally segment the graph
by using a pairwise region comparison predicate, defined in this paper.

Felzenszwalb and Huttenlocher consider two forms of graphs that an image can be
transformed into, one based only on pixel location and one that takes the color
of a pixel into account.  In the first approach, each pixel is considered a
node, and an edge is added between a pixel and its eight neighbors.  In the
second approach, the pixels are projected into a feature space, where each
pixel's features are defined to be its $(x,y)$ location in the image and its $RGB$
values.  Each pixel is then connected to its $n$ nearest neighbors in the feature
space.  This second approach allows an image to have non-contiguous segments.  In
both graph types, the edges between each node are weighted to reflect the
dissimilarity of the pixels being connected.

Two metrics are introduced to aid segmentation of these graphs, the \emph{internal
difference} ($Int(C)$) and \emph{difference between} ($Diff(C_1,C_2)$).  The
$Int(C)$ of a component $C$ is the largest weight in the minimum spanning tree
of the component.  The $Diff(C_1,C_2)$ between two components $C_1$ and $C_2$ is the
minimum weight of an edge connecting the two components, or $\infty$ if they are
not connected.  Evidence of a boundary is shown by considering the relative
difference between $Diff(C_1,C_2)$ and $max(Int(C_1),Int(C_2))$, where a large
difference is indicative of an edge.

These two metrics are used to define the concepts of segmentations that are
\emph{too fine} or \emph{too coarse}.  A segmentation that is too fine is one
in which there is no evidence for a boundary between any two segments, and a
segmentation that is too coarse is one that can be refined to a segmentation
that is not too fine.  These definitions create the existence of an optimally
segmented image which is neither too coarse nor too fine.  The segmentation
algorithm in this paper uses this definition as well as the pairwise comparison
technique to optimally segment an image, based on the author's definition of
optimal ~\cite{felzenszwalb2004}.

The algorithm developed in this paper is tested using the COIL database
~\cite{nene1996columbia} and a datast of synthetic images, and example resulting
segmentations are shown in the paper.  The algorithm is not tested against any
other algorithm, and the resulting segmentations are not compared against
human-segmented images.


\section{Research Hypothesis}
This research will seek to show that adaptively adjusting the segmentation criteria
based on the degree of variability in neighboring regions of the image improves
the segmentation over other graph-based segmentation techniques, where improving
the segmentation means that it is closer in quality to a hand-segmented image.

This is similar to the hypothesis put forth by the paper, with the addition of a
comparison test and numerical evaluation.  The lack of empirical tests left the
article's hypothesis untested.  I propose introducing a third data set,
the Berkeley Segmentation Dataset and Benchmark, which contains 500 images that
have each been hand-segmented by several people ~\cite{MartinFTM01}.  The images
from this dataset can be used to evaluate the segmentations produced by this
algorithm and will allow for comparison with other segmentation algorithms that
also use this data set.

\section{Research Plan}
In order to test this hypothesis it will be necessary to replicate not only the
algorithm proposed in this paper, but also segmentation algorithms that do not
use variability in neighboring regions as segmentation criteria.  I intend to
implement a similar algorithm that instead uses a constant threshold for inter-
region and intra-region variability for segmentation.  Additionally, I hope to
locate the results of other algorithms to compare.

Although image segmentation is a widely studied problem, it is not uncommon for
new algorithms to be put forth without an empirical evaluation.  Even when an
empirical evaluation is performed there is no standard technique used.  There
have been many evaluation techniques proposed, and I will select from these to
evaluate the algortithm proposed by \cite{felzenszwalb2004}.  For example,
several metrics for segmentation evaluation have been proposed that measure the
``goodness'' of a segmentation based on the intra-region uniformity, the
inter-region contrast, the number of mis-segmented pixels, the location of the
mis-segmented pixels, and the shape and number of the segmented regions
\cite{zhang1996survey}.  The first two techniques evaluate a segmentation without the
use of ground truth data while the others require it.

Segmentation is a hard problem even for humans.  A quick glance at the segmented
images from the Berkeley dataset reveals that different people segment the same
image in differently.  Although the overall segmentation is similar, the detail
varies greatly.  As a result, it is necessary to evaluate the variability in
human-segmented images in order to determine whether the computer segmented
image is on par.
\\
\begin{center}
\begin{tabular}{|l|l|}
\hline
\multicolumn{2}{|c|}{Timeline:}\\
\hline
Date & Goal\\
\hline
Feb. 10 & Obtain necessary datasets \\
Feb. 17 & Implement pairwise region comparison predicate \\
Feb. 24 & Implement both types of graphs \\
March 2 & Implement segmentation algorithms for comparison \\
March 9 & Implement segmentation evaluation techniques \\
March 23 & Finish testing on all data sets, consider extentions \\
April 11 & Finish presentation and paper\\
\hline
\end{tabular}
\end{center}

% XXX References
\bibliographystyle{plain}
\bibliography{sources}


\end{document}
